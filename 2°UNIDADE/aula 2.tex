
\documentclass[12pt,a4]{article}
\usepackage[utf8]{inputenc} %Pacote para acentuação
\usepackage[portuguese,brazilian]{babel}
\usepackage[lmargin=3cm,tmargin=3cm,rmargin=2cm,bmargin=2cm]{geometry} %Formato que lembra a ABNT
\usepackage[T1]{fontenc} %Ajusta o texto que vem de outras fontes
\usepackage{graphicx,xcolor,multirow,multicol}
\usepackage{amsmath,amsthm,amsfonts,amssymb,dsfont,mathtools,blindtext} %pacotes matemáticos

\begin{document}
\textbf{Árvore é um vegetal de tronco lenhoso cujos ramos só saem a certa altura do solo.[1] Em termos biológicos é uma planta permanentemente} lenhosa de grande porte, com raízes pivotantes, caule lenhoso do tipo tronco, que forma ramos bem acima do nível do solo e que se estendem até o ápice da raiz.

\begin{center}
\underline{\textbf{ arbustos, além do menor porte, podem exibir ramos desde junto ao solo. Desta maneira apenas as gimnospermas} }e angiospermas dicotiledôneas lenhosas são consideradas espécies arbóreas.[2]
\end{center}
Por pequeno porte, embora não exista uma definição consensual, costuma-se entender uma altura mínima de quatro metros na maturidade, sendo uma sequoia chamada Hyperion, localizada no Parque Nacional de Redwood ao norte de São Francisco, Estados Unidos, o maior exemplar vivo conhecido no momento, possuindo 115,55 m.[3
\end{document}
